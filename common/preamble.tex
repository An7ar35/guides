% !TeX program = LuaLaTeX
\documentclass[12pt, a4paper, oneside]{article}
\usepackage[utf8]{inputenc} 
\usepackage[left=1cm,right=1cm,top=1.5cm,bottom=1.5cm,a4paper]{geometry}
\usepackage{fontspec} %Fonts
	\setmainfont{ClearSans-Regular.ttf}[
		Path		   = fonts/clear-sans/,
		BoldFont       = ClearSans-Bold.ttf,
		ItalicFont     = ClearSans-Italic.ttf,
		BoldItalicFont = ClearSans-BoldItalic.ttf
	]
	\setmonofont{LiberationMono-Regular.ttf}[
		Path 		   = fonts/liberation-mono/,
		BoldFont	   = LiberationMono-Bold.ttf,
		ItalicFont	   = LiberationMono-Italic.ttf,
		BoldItalicFont = LiberationMono-BoldItalic.ttf
	]
	\newfontfamily\openiconic{open-iconic}[Path = fonts/]
	\newfontfamily\devicon{devicon}[Path = fonts/]
	\newcommand{\icon}[2]{\raisebox{-0.175\height}{\textcolor{#1}{\openiconic{#2}}}}
\usepackage{ragged2e} %text alignment
\usepackage[table,xcdraw,usenames,dvipsnames]{xcolor}
	\definecolor{code-background}{HTML}{f3f3f3}
	\definecolor{dark}{HTML}{31363b}
	\definecolor{linegrey}{HTML}{a8abaf}
	\definecolor{textgrey}{HTML}{777777}
\usepackage{colortbl}
\usepackage{float}
\usepackage{graphicx}
\usepackage{tabularx}
\usepackage{verbatimbox}
\usepackage{enumitem}
	\newlist{itemize-steps}{itemize}{1}
	\setlist[itemize-steps,1]{label=,leftmargin=0cm}
\usepackage{url}
\usepackage{hyperref} % Hyperlink references
	\hypersetup{colorlinks=false,pdfborder=0 0 0}
\usepackage{environ}
\usepackage{mdframed} % minipage box
	\newmdenv{allborders}
	\newmdenv[topline=false,leftline=false,rightline=false,linecolor=linegrey]{bottomborder}
	\mdfdefinestyle{commentbox}{
		topline=false,
		rightline=false,
		bottomline=false,
		linewidth=1mm,
		linecolor=linegrey,
		splittopskip=0,
		splitbottomskip=0,
		frametitleaboveskip=0,
		frametitlebelowskip=0,
		skipabove=0,
		skipbelow=0,
		leftmargin=+0.1cm,
		rightmargin=0,
		innertopmargin=2mm,
		innerbottommargin=2mm,
		roundcorner=0mm,
		backgroundcolor=white
	}
	\newmdenv[topline=false,leftline=false,rightline=false,linecolor=dark,backgroundcolor=dark,fontcolor=white]{headerborder}
\usepackage{listings} %for code
\lstset{ %
	backgroundcolor=\color{code-background},
	basicstyle=\scriptsize,
	breakatwhitespace=false,         % sets if automatic breaks should only happen at whitespace
	breaklines=true,                 % sets automatic line breaking
	commentstyle=\color{grey},   	   % comment style
	frame=single,                    % adds a frame around the code
	keepspaces=true,                 % keeps spaces in text, useful for keeping indentation of code (possibly needs columns=flexible)
	keywordstyle=\color{Violet},       % keyword style
	%numbers=left,                    % where to put the line-numbers; possible values are (none, left, right)
	%numbersep=5pt,                   % how far the line-numbers are from the code
	%numberstyle=\tiny\color{gray}, % the style that is used for the line-numbers
	%rulecolor=\color{black},         % if not set, the frame-color may be changed on line-breaks within not-black text (e.g. comments (green here))
	%stepnumber=1,                    % the step between two line-numbers. If it's 1, each line will be numbered
	tabsize=4,
	title=\lstname                   % show the filename of files included with \lstinputlisting; also try caption instead of title
}
%%% Code snippets and terminal command snippets %%%
\newcommand{\code}[1]{\small{\colorbox{code-background}{\texttt{#1}}}} %Code text formatting
\newcommand{\terminalcmd}[1]{\textcolor{orange}{\$} \code{#1}} %Terminal line formatting
%%% Blocks %%%
\newcommand{\blockheader}[1]{\raggedright{\textbf{\textcolor{textgrey}{#1}}\linebreak}}
\newcommand{\block}[1]{\begin{mdframed}[style=commentbox]\color{dark}{#1}\end{mdframed}}
\newenvironment{blocksection}{\begin{mdframed}[style=commentbox]\color{dark}}{\end{mdframed}}

%\usepackage{showframe}