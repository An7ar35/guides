% This text is released under the Creative Commons Attribution-NonCommercial-ShareAlike 4.0 International (CC BY-NC-SA 4.0) License
% Full license available at https://creativecommons.org/licenses/by-nc-sa/4.0/legalcode
\documentclass[12pt,a4paper,oneside]{article}
\usepackage[left=0cm,right=0cm,top=0cm,bottom=0cm,a4paper]{geometry}
\usepackage{import}
	\import{../common/}{fonts.tex}
	\import{../common/}{graphics.tex}
	\setlength{\ULdepth}{5pt}
\usepackage[none]{hyphenat}
\usepackage[table,xcdraw,usenames,dvipsnames]{xcolor}
	\definecolor{code-background}{HTML}{f3f3f3}
	\definecolor{dark}{HTML}{31363b}
	\definecolor{linegrey}{HTML}{a8abaf}
	\definecolor{textgrey}{HTML}{777777}
	\definecolor{codekeyword1}{HTML}{3aa3ff}
\usepackage{colortbl}
\usepackage{float}
\usepackage{caption}
\usepackage{tabularx}
\newcolumntype{R}{>{\raggedleft\let\newline\\\arraybackslash\hspace{0pt}}X}
\usepackage{verbatimbox}
\usepackage{soul}
	\sethlcolor{code-background}
\usepackage[skins]{tcolorbox}
	\tcbset{commonstyle/.style={boxrule=0pt,sharp corners,enhanced,nobeforeafter,boxsep=0pt,colback=code-background,fit width from 0 to \textwidth}}
	\newtcolorbox{codesnippetbox}[1][]{commonstyle,#1}
\usepackage{enumitem}
\usepackage{varwidth}
\usepackage[bottom]{footmisc}
\usepackage{url}
\usepackage{hyperref} % Hyperlink references
\hypersetup{colorlinks=true,linkcolor=dark,urlcolor=MidnightBlue,pdfborder=0 0 0}
\usepackage{environ}
\usepackage[pages=some]{background}
%\usepackage{memoir}
\usepackage{mdframed} % minipage box
	\newmdenv{allborders}
	\newmdenv[topline=false,leftline=false,rightline=false,linecolor=linegrey]{bottomborder}
	\newmdenv[topline=false,leftline=false,rightline=false,linecolor=dark,backgroundcolor=dark,fontcolor=white]{headerborder}
	\mdfdefinestyle{titlebox}{
		leftline=false,
		rightline=false,
		innertopmargin=2mm,
		innerbottommargin=2mm,
		linewidth=1pt,
		backgroundcolor=dark, 
		fontcolor=white, 
		linecolor=white
	}
	\mdfdefinestyle{commentbox}{
		topline=false,
		rightline=false,
		bottomline=false,
		linewidth=1mm,
		linecolor=linegrey,
		splittopskip=0,
		splitbottomskip=0,
		frametitleaboveskip=0,
		frametitlebelowskip=0,
		skipabove=0,
		skipbelow=0,
		leftmargin=+0.3cm,
		rightmargin=0,
		innertopmargin=2mm,
		innerbottommargin=2mm,
		roundcorner=0mm,
		backgroundcolor=white
	}
	\mdfdefinestyle{codebox}{
		topline=false,
		rightline=false,
		leftline=false,
		bottomline=false,
		splittopskip=0,
		splitbottomskip=0,
		frametitleaboveskip=0,
		frametitlebelowskip=0,
		skipabove=0,
		skipbelow=0,
		leftmargin=0,
		rightmargin=0,
		innertopmargin=2mm,
		innerbottommargin=2mm,
		roundcorner=0mm,
		backgroundcolor=code-background
	}
\usepackage{tikz}
	\usetikzlibrary{shadows}
	\usetikzlibrary{shapes.geometric}
\usepackage{listings} %for code
	\lstset{ %
		backgroundcolor=\color{code-background},
		basicstyle=\scriptsize,
		breakatwhitespace=false,         % sets if automatic breaks should only happen at whitespace
		breaklines=true,                 % sets automatic line breaking
		commentstyle=\color{grey},   	   % comment style
		frame=single,                    % adds a frame around the code
		keepspaces=true,                 % keeps spaces in text, useful for keeping indentation of code (possibly needs columns=flexible)
		keywordstyle=\color{Violet},       % keyword style
		%numbers=left,                    % where to put the line-numbers; possible values are (none, left, right)
		%numbersep=5pt,                   % how far the line-numbers are from the code
		%numberstyle=\tiny\color{gray}, % the style that is used for the line-numbers
		%rulecolor=\color{black},         % if not set, the frame-color may be changed on line-breaks within not-black text (e.g. comments (green here))
		%stepnumber=1,                    % the step between two line-numbers. If it's 1, each line will be numbered
		tabsize=4,
		title=\lstname                   % show the filename of files included with \lstinputlisting; also try caption instead of title
	}
%%% Header %%%
\newenvironment{headersection}{\begin{headerborder}\vspace{0.5cm}\begin{minipage}[t]{\textwidth}}{\end{minipage}\vspace{0.5em}\end{headerborder}}
%%% Code snippets and terminal command snippets %%%
\newcommand{\code}[1]{\begin{small}\texttt{\sethlcolor{code-background}\hl{#1}}\end{small}} %Code text formatting
\newcommand{\terminalcmd}[1]{\begin{tabularx}{\textwidth}{cX}\textcolor{orange}{\$} & \code{#1}\end{tabularx}} %Terminal line formatting
%%% Blocks %%%
\newcommand{\blockheader}[1]{\raggedright{\textbf{\textcolor{textgrey}{#1}}\linebreak}}
\newcommand{\block}[1]{\begin{mdframed}[style=commentbox]\color{dark}{#1}\end{mdframed}}
\newenvironment{blocksection}{\begin{mdframed}[style=commentbox]\color{dark}}{\end{mdframed}}
\newenvironment{codeblock}{\begin{mdframed}[style=codebox]\color{dark}\small\ttfamily}{\par\end{mdframed}}
%%% Custom hrefs %%%
\newcommand{\cref}[3]{\href{#2}{\color{#1}{#3}}}%
%%% Custom free bullet point
\newcommand{\cbullet}[1]{\hangindent=0.57em{· #1}}
%%% Custom keyboard key box %%%
\newcommand*\key[1]{
	\tikz[baseline=(key.base)]
	\node[draw,
		fill=white,
		drop shadow={
			shadow xshift=0.25ex, shadow yshift=-0.25ex, fill=dark, opacity=0.75
		},
		regular polygon,
		regular polygon sides=4,
		rounded corners=2pt,
		inner sep=1pt,
		line width=1pt,
		minimum size=1pt,
		scale=0.72
		](key) {\textcolor{textgrey}{\dejasans{#1}}\strut};
}
\newcommand\tab[1][1cm]{\hspace*{#1}}
%\usepackage{showframe}

\color{dark}
\backgroundsetup{
	scale=1,
	color=dark,
	opacity=1,
	angle=0,
	contents={%
		\includegraphics[width=\paperwidth,height=\paperheight]{img/Burst.pdf}
	}%
}


\begin{document}

\pagecolor{dark}
\color{white}
%\BgThispage
\sloppy
\centering
\vspace*{0.5em}
\begin{tabularx}{0.95\linewidth}{lR}
	\cref{white}{https://an7ar35.bitbucket.io/}{\raisebox{-0.25\height}{\includegraphics[scale=0.25]{../common/graphics/avatar.png}} An7ar35} & 
	December 2017
\end{tabularx}

\vspace*{\fill}
\includegraphics[height=10em]{img/ArchMBP.pdf}\linebreak
\begin{mdframed}[style=titlebox]
	\centering
	\begin{Huge}
		\industrial{Arch Linux installation guide\linebreak
			for\linebreak
			MacBook Pro Retina mid 2015 model}\par
	\end{Huge}
\end{mdframed}
\vspace*{2em}
\includegraphics[height=5em]{img/kde-logo.pdf}with KDE\par
\vspace*{\fill}
\begin{tabularx}{0.95\textwidth}{lR}
	& \includegraphics[height=3em]{../common/licenses/by-nc-sa_eu.pdf}
\end{tabularx}
\vspace*{2em}

\newgeometry{left=1cm,right=1cm,top=1cm,bottom=1.5cm}
\setcounter{page}{1}
\pagecolor{white}
\color{dark}
\normalsize\justify
\tableofcontents
\clearpage

\clearpage
\section{Forewords}

After lots of reading, searching, experimenting, furious late-night shell command typing and do-overs here are the results of replacing OSX with Arch Linux (KDE) on a mid-2015 MacBook Pro. 

I would also advise backing up your drive using a complete bit-to-bit cloning process (unlike me who casually forgot that step...) so that you retain a copy of everything including the recovery partition on the Mac. This way it is a one-step process to restore everything.

This guide assumes a basic working knowledge of Linux command line (bash) as well as Arch's \href{https://wiki.archlinux.org/index.php/Pacman}{\code{pacman}} and \href{https://archlinux.fr/yaourt-en}{\code{yaourt}\textsuperscript{AUR}} package managers.

\section{Nomenclatures}

\begin{tabularx}{\textwidth}{lX}
	\code{package-name} & Standard package \href{https://www.archlinux.org/packages/}{repository} - use \code{pacman} to install.\\
	\code{package-name}\textsuperscript{AUR} & Arch User Repository package (\href{https://aur.archlinux.org/}{AUR}) - use \code{yaourt} to install.\\
	\textcolor{orange}{\$} \code{...} & Command to type in at the command line prompt.\\
	\textcolor{codekeyword1}{\$TEXT} & User defined variable (replace with what you want to define it as).\\
	\fbox{Key} & Keyboard key
\end{tabularx}

\clearpage
\section{Machine Specs}

\begin{tabularx}{\textwidth}{|c|X|}
	\hline
	Display   & \cbullet{15.4" LED-backlit Retina display (2880x1800 at 220ppi)} \\\hline
	Processor & \cbullet{2.5GHz quad-core Intel Core i7 processor (Turbo Boost up to 3.7GHz) with 6MB shared L3 cache} \\\hline
	RAM       & · 16GB of 1600MHz DDR3L memory \\\hline
	GPU       & · Intel Iris Pro Graphics, \newline
	· AMD Radeon R9 M370X with 2GB of GDDR5 memory\\\hline
	Storage   & · 512GB PCIe-based flash \\\hline
	Webcam    & · 720p FaceTime HD camera \\\hline
	Network   & · 802.11ac Wi‑Fi wireless networking; IEEE 802.11a/b/g/n compatible, \newline
	· Bluetooth 4.0 wireless technology \\
	\hline
\end{tabularx}

\section{Results}

\begin{small}\fcolorbox{linegrey}{white}{
		\textcolor{textgrey}{\textbf{Legend: }}
		\raisebox{-0.2\height}{\color{green}{\openiconic[]}} Works out-of-the-box or with few steps, 
		\raisebox{-0.2\height}{\color{blue}{\openiconic[]}} Requires some work, 
		\raisebox{-0.2\height}{\color{orange}{\openiconic[]}} Not fully working, 
		\raisebox{-0.2\height}{\color{red}{\openiconic[]}} Broken.}
\end{small}

\begin{center}
	\setlength\arrayrulewidth{1pt}
	\rowcolors{2}{gray!25}{white}
	\begin{tabular}{lcl}
		\rowcolor{white!50}
		\textbf{Item} & \textbf{Result} & \textbf{Note}\\
		\hline\hline
		Wifi\footnotemark[1] & \raisebox{-0.2\height}{\color{green}{\openiconic[]}} & Works out-of-the-box, some error messages.\\
		Bluetooth & \raisebox{-0.2\height}{\color{green}{\openiconic[]}} & Uses standard \code{bluez} package.\\
		Display (IPS scaling)\footnotemark[2] & \raisebox{-0.2\height}{\color{orange}{\openiconic[]}} & Mixed results without Wayland\\
		Display (Backlight) & \raisebox{-0.2\height}{\color{blue}{\openiconic[]}} & Needs a kernel patch\footnotemark[3] for \code{apple-gmux}.\\
		AMD Radeon R9 M370X & \raisebox{-0.2\height}{\color{green}{\openiconic[]}} & Open source driver works like a charm\\
		Intel Iris Pro Graphics\footnotemark[4] & \raisebox{-0.2\height}{\color{red}{\openiconic[]}} & Disabled by default, graphic switching not working.\\
		Audio & \raisebox{-0.2\height}{\color{green}{\openiconic[]}} & Uses ALSA.\\
		FaceTimeHD camera\footnotemark[5] & \raisebox{-0.2\height}{\color{orange}{\openiconic[]}} & Requires firmware and driver from AUR. Suspend issues.\\
		Keyboard & \raisebox{-0.2\height}{\color{green}{\openiconic[]}} & ISO layout in command line but perfect in KDE.\\
		Keyboard (Backlight) & \raisebox{-0.2\height}{\color{green}{\openiconic[]}} & Works out-of-the-box.\\
		Keyboard \fbox{Fn} keys & \raisebox{-0.2\height}{\color{green}{\openiconic[]}} & Works out-of-the-box in KDE.
	\end{tabular}
\end{center}

\footnotetext[1]{Error messages with default kernel module but doesn't seem to affect connections.}
\footnotetext[2]{A workable solution can be setting the resolution to 1920x1200. The scale is good for a 15" screen.}
\footnotetext[3]{Patch needs to be re-applied after any kernel updates at the moment.}
\footnotetext[4]{Being stuck with using the dedicated GPU means that the battery will drain faster than when using Intel's IGU.}
\footnotetext[5]{Not all applications seem to work with this camera but it might just be compatibility problems on the app side.}

\clearpage
\section{Pre-Installation}

\subsection{Preparations}

\textbf{Backup drive} (complete drive clone preferable).

Make a copy of the colour profile file(s) on the mac to a USB stick. It will be useful later on Linux. The profiles are located in \code{/Library/ColorSync/Profiles/Displays/*}.

\subsection{Making a bootable USB}

\subsubsection{From Linux}

\terminalcmd{dd if=archlinux.iso of=/dev/sdX bs=16M \&\& sync}
\block{Where \code{X} is your target USB drive letter (use \code{lsblk} for an overview of all connected drives to find out)}

\subsubsection{From a Mac}

\terminalcmd{diskutil unmountDisk diskX}
\block{Where \code{X} is your target USB drive number (use \code{diskutil list} for an overview of all connected drives to find out).}
\terminalcmd{sudo dd if=/Users/\$USERNAME/Downloads/archlinux.iso of=/dev/diskX bs=16M}
\block{Replace \textcolor{codekeyword1}{\$USERNAME} with your user-name on the mac and replace \code{X} with the USB drive's number.}

\section{Installation}

\subsection{Booting from the USB stick}

Simply plug in the USB in the MBP and press the \fbox{Alt} (a.k.a. \fbox{options}) key during start up
to reach the boot menu.

\subsection{Preliminary setup}

\terminalcmd{loadkeys uk}
\block{Load your keyboard layout. Replace `uk` with whichever you have on your machine.}
\terminalcmd{wifi-menu}
\block{Connect up to the wifi network. On the MPB Pro 2015 the wifi gets detected and works out of the box at this stage.}

\subsection{Partitioning and Crypto volume setup}

The assumption here is that the hard drive where everything will be installed to is \code{/dev/sda}. The partition structure will thus be: \\
\code{/dev/sda1} for the EFI partition,\\
\code{/dev/sda2} for the Boot partition and\\
\code{/dev/sda3} for the encrypted volume where the root, home and swap will be located.\\

\terminalcmd{cgdisk /dev/sda}
\begin{blocksection}
	\blockheader{Partition layout with just Linux}
	\textit{Warning: this will erase every thing including the recovery partition for OSX. \\
		No easy way to go back after this without a bit-to-bit HDD clone}
	\begin{enumerate}
		\item 100M partition `EFI' (Hex \#\code{ef00})
		\item 250M partition `Boot' (Hex \#\code{8300})
		\item 100\% remainder of the space for the crypto volume (Hex \#\code{8300})
	\end{enumerate}
	The root and home partition will be created later in the crypto volume
\end{blocksection}
\terminalcmd{cryptsetup --verbose --verify-passphrase --cipher aes-xts-plain64 --key-size 512 --hash sha512 --iter-time 5000 --use-random luksFormat /dev/sda3}
\block{Sets up the crypto volume.}
\terminalcmd{cryptsetup open --type luks /dev/sda3 mctoasty}
\block{\code{mctoasty} is the mounted name of the crypto volume. Can be changed to whatever is preferred.}

\subsubsection{Create crypto volume partitions}

\terminalcmd{pvcreate /dev/mapper/mctoasty}
\terminalcmd{vgcreate vg0 /dev/mapper/mctoasty}
\terminalcmd{lvcreate --size 16G vg0 --name swap}
\block{Generally swap size value is set to the amount of RAM.}
\terminalcmd{lvcreate --size 50GB vg0 --name root}
\block{Adjust root's size as required.}
\item \terminalcmd{lvcreate -l +100\%FREE vg0 --name home}
\block{Creates the home parition with the remainder of the free space.}

\subsubsection{Formatting all the partitions}

\terminalcmd{mkfs.vfat -F32 /dev/sda1}
\block{EFI partition}
\terminalcmd{mkfs.ext2 /dev/sda2}
\block{Boot partition}
\terminalcmd{mkswap /dev/mapper/vg0-swap}
\terminalcmd{mkfs.ext4 /dev/mapper/vg0-root}
\terminalcmd{mkfs.ext4 /dev/mapper/vg0-home}

\subsubsection{Mount all the partitions}

\terminalcmd{mount /dev/mapper/vg0-root /mnt}
\terminalcmd{swapon /dev/mapper/vg0-swap}
\terminalcmd{mkdir /mnt/boot}
\terminalcmd{mount /dev/sda2 /mnt/boot}
\terminalcmd{mkdir /mnt/boot/efi}
\terminalcmd{mount /dev/sda1 /mnt/boot/efi}
\terminalcmd{mkdir /mnt/home}
\terminalcmd{mount /dev/mapper/vg0-home /mnt/home}

\section{System Setup I: Base System Installation}

\terminalcmd{pacstrap /mnt base base-devel grub-efi-x86\_64 git efibootmgr bash-completion dialog wpa\_supplicant}
\begin{blocksection}
	Base packages along with GRUB and the EFI boot manager, git (will be useful later), bash completion and the stuff needed to keep the Wifi device in working order after reboot.
\end{blocksection}

\subsection{Generating fstab}

\terminalcmd{genfstab -pU /mnt >> /mnt/etc/fstab}
\begin{blocksection}
	\blockheader{Important!}
	This generates the fstab. i.e.: It saves our mounted partitions and swap configuration for persistence after a reboot. If you miss this step you'll have to reboot with the USB, mount everything again and then generate the fstab file.
\end{blocksection}
\terminalcmd{arch-chroot /mnt /bin/bash}

\subsection{System clock}

\terminalcmd{ln -s /usr/share/zoneinfo/\$ZONE/\$REGION /etc/localtime}
\begin{blocksection}
	Replace \textcolor{codekeyword1}{\$ZONE} with yours from \code{/usr/share/zoneinfo/}\\
	Replace \textcolor{codekeyword1}{\$REGION} with yours from \code{/usr/share/zoneinfo/\$ZONE/}\\
	e.g.: \code{/usr/share/zoneinfo/Europe/London}
\end{blocksection}
\terminalcmd{hwclock --systohc --utc}

\subsection{Hostname}

\terminalcmd{echo \$HOSTNAME > /etc/hostname}
\terminalcmd{nano /etc/hosts}
\begin{blocksection}
	Add the hostname at the end of each of the relevant lines in this file for completion's sake. Replace \textcolor{codekeyword1}{\$HOSTNAME} by the hostname you want to computer to have.
\end{blocksection}

\subsection{Basic fonts}

\terminalcmd{pacman -S terminus-font ttf-dejavu ttf-liberation}
\begin{blocksection}
	The terminus font will be used to make the console font more \href{https://wiki.archlinux.org/index.php/HiDPI#Linux_console}{readable} on the HiDPI display.
\end{blocksection}

\subsection{Locale}

\terminalcmd{locale-gen}
\block{Generates the locale file}
\terminalcmd{nano /etc/local.conf}
\block{Edit the locale configuration and delete the hash in front of the desired locale.}
\terminalcmd{local-gen}
\block{Needs to run again to apply the changes made in the previous step.}
\terminalcmd{nano /etc/locale.conf}
\begin{blocksection}
	To set permanent locale settings add the lines:\\
	\begin{codeblock}
		LANG=en\_GB.UTF-8\\
		LANGUAGE=en\_GB\\
		LC\_ALL=C
	\end{codeblock}
\end{blocksection}

\subsection{Console keymap and font}

To see a list of all available keymaps: \code{find /usr/share/kbd/keymaps/ -type f | more}\\
To see a list of all installed console fonts: \code{ls /usr/share/kbd/consolefonts/}\\
For font maps check the \href{https://wiki.archlinux.org/index.php/fonts#Persistent_configuration}{Arch Wiki} and the \href{https://en.wikipedia.org/wiki/ISO/IEC_8859#The_parts_of_ISO/IEC_8859}{Wikipedia} entries.

\terminalcmd{nano /etc/vconsole.conf}
\begin{blocksection}
	To set permanent setting for the console add the lines:\\
	\begin{codeblock}
		KEYMAP=uk\\
		FONT=ter-228n\\
		FONT\_MAP=8859-1
	\end{codeblock}
\end{blocksection}

\subsection{Root/User Accounts}

\terminalcmd{passwd}
\block{Sets the root password.}
\terminalcmd{useradd -m -g users -G wheel -s /bin/bash \$USERNAME}
\block{Adds a user. Replace \textcolor{codekeyword1}{\$USERNAME} with whatever user name you which to use.}
\terminalcmd{passwd \$USERNAME}
\block{Sets the user's password.}

\subsection{Initial RAM Environment Configuration [\href{https://wiki.archlinux.org/index.php/mkinitcpio}{\texttt{mkinitcpio}}]}

\terminalcmd{nano /etc/mkinitcpio.conf}
\begin{blocksection}
	In the file do the following:
	\begin{itemize}[noitemsep,topsep=0pt,leftmargin=*]
		\item In 'MODULES:
		\begin{enumerate}
			\item Add `ext4`
		\end{enumerate}
		\item In 'HOOKS':
		\begin{enumerate}
			\item Add \code{encrypt} and \code{lvm2} before \code{filesystems}
			\item Add \code{consolefont} right after \code{autodetect}\\
			(to avoid squinting at the crypto volume password prompt)
			\item Move \code{keyboard} right after \code{consolefont}
		\end{enumerate}
	\end{itemize}
	HOOKS should be ordered as such in the end:\\
	\code{HOOKS=(base udev autodetect consolefont keyboard modconf block encrypt lvm2 filesystems fsck)}
\end{blocksection}

\terminalcmd{mkinitcpio -p linux}
\block{Regenerates the initrd image.}

\subsection{GRUB boot loader}

\terminalcmd{grub-install}
\terminalcmd{nano /etc/default/grub}
\begin{blocksection}
	At line with \code{GRUB\_CMDLINE\_LINUX} add the arguments so that it becomes\\
	\code{GRUB\_CMDLINE\_LINUX="cryptdevice=/dev/sda3:luks:allow-discards"}
\end{blocksection}
\terminalcmd{grub-mkconfig -o /boot/grub/grub.cfg}

\subsection{Dismount and reboot}

\terminalcmd{umount -R /mnt}
\terminalcmd{swapoff -a}

Remove the USB stick then \code{reboot}.

\textbf{Note}: If you want a break this is the place to take it. Instead of rebooting just shutdown the machine.

\section{System Setup II: Hardware and Tools}

Reconnect to your wifi with \code{sudo wifi-menu}

\subsection{Enabling the multilib package repository}

\terminalcmd{nano /etc/pacman.conf}
\begin{blocksection}
	Uncomment the multilib lines (i.e. remove the leading \code{\#}):
	\begin{codeblock}
		\#[multilib]\\
		\#Include = /etc/pacman.d/mirrorlist
	\end{codeblock}
\end{blocksection}
\terminalcmd{pacman -Syuu}
\block{Updates the cache}

\subsection{AUR package manager}

From your home directory (\code{cd \textasciitilde}):

\terminalcmd{mkdir -p git-repos/system-packages}
\terminalcmd{cd git-repos/system-packages}
\block{Or whatever directory chosen to clone the repositories into.}
\terminalcmd{git clone https://aur.archlinux.org/package-query.git}
\block{Required for \href{https://github.com/archlinuxfr/yaourt}{Yaourt}}
\terminalcmd{git clone https://aur.archlinux.org/yaourt.git}
\block{Easy AUR package installations from the console.}

For each of the 2 packages, \code{cd} into their respective directories and run the following command to install:

\terminalcmd{makepkg -si}
\block{e.g.: \code{cd package-query} then \code{makepkg -si}}

\subsection{Console-Fu}

\terminalcmd{yaourt -S hstr-git}
\block{This is a replacement on \href{https://github.com/dvorka/hstr}{steroids} for the console's \fbox{ctrl}+\fbox{r}}
\terminalcmd{hh --show-configuration >> ~/.bashrc}
\block{Adds the config options for HSTR to your bash profile and auto-starts hh on login.}
\terminalcmd{pacman -S powertop htop}
\block{Installs 2 of the most basic and useful monitoring tools in linux.}

\subsection{Power management}

\terminalcmd{yaourt -S laptop-mode-tools}
\block{All the laptop-centric \href{https://wiki.archlinux.org/index.php/Laptop_Mode_Tools}{power saving} goodies}
\terminalcmd{sudo systemctl enable laptop-mode.service}
\block{Turns on the service} 
\terminalcmd{pacman -S acpid}
\block{\emph{(Optional)} \href{https://wiki.archlinux.org/index.php/Acpid}{Daemon} for delivering ACPI events.}

\subsection{Hardware}

\subsubsection{Fans}

\terminalcmd{yaourt -S mbpfan-git}
	\block{\href{https://github.com/dgraziotin/mbpfan}{mbpfan-git}\textsuperscript{AUR} controls the MacBook's fans.}
\terminalcmd{sudo systemctl enable mbpfan.service}
	\block{Turns on the service}

The configuration file is located in \code{/etc/mbpfan.conf}. Go there to customise the temperature limits and fan speeds.

\subsubsection{Processor}

\terminalcmd{pacman -S intel-ucode}
\block{For intel's \href{https://wiki.archlinux.org/index.php/microcode}{microcode} update.}
\terminalcmd{grub-mkconfig -o /boot/grub/grub.cfg}
\block{Need to update grub so that it loads the Microcode updates at boot}

\subsubsection{Processor: Turbo Boost}

I would recommend disabling this feature if usage includes sustained load-heavy computation such as rendering, compiling, etc...

Turbo boost automatically increases the operating frequency of the cores depending on the task load. This allows for greater performance under demanding conditions but also causes heating issues. These are exasperated by the design constraint of laptops where large heat-sinks + fans or water-cooling is not practical.

Turning off Intel's CPU turbo boost makes a sizeable difference to temperature\footnote{Odly enough, I found that on my MBP the temperatures reached on OSX are a little worse than on Linux.} both on idle and under load. A cooler processor will also help increase the lifespan of the machine.

The table below shows the different temperature reached with and without the turbo boost enabled on my machine. The load was from compiling a small C++/Qt application.
The maximum turbo boost temperature was reached 10 seconds inside the compiling job and stayed at $100\,^{\circ}\mathrm{C}$ until completion with the fans going full blast.

\begin{center}
	\vspace*{1em}
	\rowcolors{2}{gray!25}{white}
	\setlength\arrayrulewidth{1pt}
	\begin{tabular}{|l|c|c|}
		\rowcolor{white!50}
		\hline
		\textbf{Turbo Boost state} & \textbf{Idle Temp.} & \textbf{Load Temp.}\\
		\hline\hline
		Enabled & $60\to 65\,^{\circ}\mathrm{C}$ & $100\,^{\circ}\mathrm{C}$\\ 
		Disabled &  $48\to 50\,^{\circ}\mathrm{C}$ & $76\,^{\circ}\mathrm{C}$\\
		\hline
	\end{tabular}
	\vspace*{1em}
\end{center}

\textbf{\textcolor{textgrey}{\underline{Requirements}}}

\href{https://01.org/msr-tools}{MSR-Tools}\textsuperscript{AUR} provides utilities to access the processor MSRs and CPU ID directly. It is called by the turbo-boost disabler script.

\terminalcmd{yaourt -S msr-tools}

\vspace*{1em}
\textbf{\textcolor{textgrey}{\underline{Disabling Turbo Boost}}}

Clone the \href{https://github.com/An7ar35/arch-scripts}{arch-scripts} repository into the \code{~/git-repos/system-packages} directory previously created.

\terminalcmd{cd ~/git-repos/system-packages}
\terminalcmd{git clone https://github.com/An7ar35/arch-scripts}
\terminalcmd{cd arch-scripts/coreboost/}
\terminalcmd{sudo ./install.sh}
\begin{blocksection}
	The installation copies the \code{coreboost.sh} script into \code{/usr/local/bin/} and the systemd service file to launch that script into \code{/etc/systemd/system/}.\newline
	It then enables the service and loads it up. The turbo-boost disabler will be called upon at boot and after resuming suspend automatically.
\end{blocksection}

\subsubsection{Sound}

\href{https://wiki.archlinux.org/index.php/ALSA}{Alsa} works without issues.

\terminalcmd{sudo pacman -S alsa-utils alsa-plugins}
\terminalcmd{alsamixer}
\begin{blocksection}
	Make sure your current sound card is the "HDA Intel PCH" and that your master volume is up and unmuted (mute=MM, unmuted=00 at the bottom of the volume bar. You can use the \fbox{M} key on the keyboard to toggle mute).
\end{blocksection}
\terminalcmd{speaker-test -c 2}
\block{To make sure the sound works.}

\textbf{Note}: The internal speaker might not be disabled when using the headphone jack. To solve this, enable "Auto-mute" in \code{alsamixer}.

\subsubsection{Bluetooth}

\href{https://wiki.archlinux.org/index.php/Bluetooth}{Bluetooth} works out-of-the-box with the standard packages.

\terminalcmd{sudo pacman -S bluez bluez-utils}
\terminalcmd{modprobe btusb}
\terminalcmd{sudo systemctl enable bluetooth.service}

\subsubsection{Video}
\terminalcmd{sudo pacman -S mesa xf86-video-amdgpu vulkan-radeon lib32-mesa}
\begin{blocksection}
	Installs the open source drivers for the ATI GPU along with the 32bit libs and the Vulkan drivers.\\
	The \href{https://wiki.archlinux.org/index.php/Lm_sensors}{\code{lm\_sensors}} package used for temperature monitoring is a dependency for \code{mesa} so will be installed with it.\\
	Run \code{sensors} to see all the temperatures.
\end{blocksection}
\terminalcmd{yaourt -S radeontop}
\block{\emph{(Optional)} Monitoring utility for Radeon GPU cards.}

\subsubsection{Display brightness}

To get brightness control to work, a patched version of the \code{apple-gmux} kernel module is required. The vanilla module does \href{https://bugzilla.kernel.org/show_bug.cgi?id=105051#c37}{not work} as of Linux kernel version 4.14.7-1. Perhaps later versions will eventually.

An easy installer script is available in the \href{https://github.com/An7ar35/arch-scripts}{arch-script repository}.

If you haven't previously cloned the repository (Turbo Boost section):

\terminalcmd{cd ~/git-repos/system-packages}
\terminalcmd{git clone https://github.com/An7ar35/arch-scripts}

\vspace*{1em}
\textbf{\textcolor{textgrey}{\underline{Installing the patched module}}}

\terminalcmd{cd arch-scripts/mbp-brightness-patch/}
\terminalcmd{sudo ./install.sh}

Then restart the machine.

\vspace*{1em}
\textbf{\textcolor{textgrey}{\underline{Kernel updates}}}

If the brightness control breaks after a kernel update the patch must be run again to get the functionality back.

\subsubsection{Keyboard}

All of these tweaks are optional and based on personal choice.

\vspace*{1em}
\textbf{\textcolor{textgrey}{\underline{Switch function keys on by default}}}

To make function keys be used as first keys. i.e.: Pressing \fbox{F1} key alone will behave like F1 and pressing \fbox{fn}+\fbox{F1} will act as special key (brightness down).

\terminalcmd{sudo nano /etc/modprobe.d/hid\_apple.conf}
\begin{blocksection}
	Add the following line to the file:
	\begin{codeblock}
		options hid\_apple fnmode=2
	\end{codeblock}
\end{blocksection}
\terminalcmd{sudo mkinitcpio -p linux}
	\block{Updates the \textit{initramfs} with the new configuration.}

\subsubsection{Webcam}

The 2015 MBP has \href{https://wiki.archlinux.org/index.php/MacBook#Facetime_HD}{Facetime HD}.
Fortunately there is a \href{https://github.com/patjak/bcwc_pcie}{reversed-engineered driver} but "PC suspension is not supported if a 
program that is keeping the camera active is running".

To make it work, first install the firmware then the driver from AUR:

\terminalcmd{yaourt -S facetimehd-firmware}
\block{Installs the firmware alone.}
\terminalcmd{yaourt -S bcwc-pcie-git}
\block{Gets the driver.}

To test the webcam, \code{mplayer tv://} or the \lq \textit{Qt 4VL Test Utility}\rq\ can be used inside KDE.

\begin{table}[!h]
	\centering
	\vspace*{1em}
	\rowcolors{2}{gray!25}{white}
	\setlength\arrayrulewidth{1pt}
	\caption*{Tested applications} \label{tab:tested-webcam-apps} 
	\begin{tabular}{|l|l|c|}
		\rowcolor{white!50}
		\hline
		\textbf{App} & \textbf{Version} & \textbf{Works?} \\
		\hline\hline
		Discord & 0.0.3-1 & \ding{51}\\
		Skype & 8.13.76.6-1 & \ding{55}\\
		Kamoso & 3.2.4-1 & \ding{55}\\
		\hline
	\end{tabular}
	\vspace*{1em}
\end{table}

\subsubsection{Touchpad}

Basic touchpad support is available with the Linux kernel. If you want to customise your experience, a specialised synaptic driver must be installed. There are 2 main options:

\vspace*{1em}
\textbf{\textcolor{textgrey}{\underline{Option 1: \code{xf86-input-synaptics}}}}

The standard synaptic driver. KDE will detect it and enable you to change its settings within the \lq System Settings\rq\ panel. Only supports 2 finger gesture though.

\terminalcmd{sudo pacman -S xf86-input-synaptics}

\vspace*{1em}
\textbf{\textcolor{textgrey}{\underline{Option 2: \code{xf86-input-mtrack}\textsuperscript{AUR}}}}

Multi-touch support but KDE does \textbf{not} natively detect it so settings must be set manually in \code{/etc/X11/xorg.conf.d/10-mtrack.conf}. A full list of supported options is available in the project's Github \href{https://github.com/p2rkw/xf86-input-mtrack}{repository}.

\terminalcmd{yaourt -S xf86-input-mtrack}

\subsubsection{Other Drivers}

Possible missing firmware module for:
\begin{itemize}[noitemsep,topsep=0pt,leftmargin=*]
	\item wd719x
	\item aic94xx
\end{itemize}

\terminalcmd{pacman -S aic94xx-firmware}
\terminalcmd{pacman -S wd719x-firmware}
\terminalcmd{sudo mkinitcpio -p linux}
\block{To make sure drivers are loaded.}

\section{System Setup III: Desktop}

Wayland support as of Dec 2017 on KDE is beta at best. Scaling in nice but buggy as hell displaying artefacts. Until that gets a lot better Xorg is the default choice as display managers go even if HiDPI support is terrible. 

\subsection{Display server (\href{https://wiki.archlinux.org/index.php/xorg}{Xorg})}

\terminalcmd{sudo pacman -S xorg, xorg-server xorg-xinit}

\subsection{Desktop Environment (\href{https://wiki.archlinux.org/index.php/KDE}{KDE})}

A display manager (\href{https://wiki.archlinux.org/index.php/SDDM}{SDDM}) is installed along with KDE to make it easy to automatically run KDE when starting a user session and provide a GUI login too.

\terminalcmd{sudo pacman -S plasma kde-applications sddm systemd-kcm}
\terminalcmd{sudo systemctl enable sddm.service}
\terminalcmd{sudo systemctl start sddm.service}

Auto-login can be configured inside KDE's \lq System Settings\rq\ panel (\lq Startup \& Shutdown\rq\ \rightarrow\ \lq Login Screen (SDDM)\rq\ \rightarrow\ \lq Advanced\rq\ tab).

\terminalcmd{sudo pacman -S xdg-user-dirs}
\begin{blocksection}
	\blockheader{Optional} 
	Create all the default \href{https://wiki.archlinux.org/index.php/XDG_user_directories}{user directories} such as \lq Downloads\rq\ , \lq Music\rq\  , etc...
\end{blocksection}

\subsection{Network manager (Wifi)}
\terminalcmd{sudo systemctl enable NetworkManager.service}
\block{Enables the network service. GUI network connections should work after reboot.}

\subsection{GUI package manager}

\href{https://wiki.manjaro.org/index.php?title=Pamac}{Pamac} is a nice package manager with AUR support. It can be installed with:

\terminalcmd{yaourt -S pamac-aur}
\block{AUR support can be explicitly enabled inside the application's settings in the AUR tab.}

\subsection{Colour Profile}

//TODO

\subsection{Printing/Scanning}

\terminalcmd{sudo pacman -S cups print-manager}
\terminalcmd{//TODO scanning}

\section{System Setup IV: Applications}

`flite` (missing packages causes a "text-to-speech" error - see `journalctl -b`)


\subsection{File system}

\terminalcmd{sudo pacman -S ntfs-3g exfat-utils}
\block{Enables access to NTFS and ExFAT file systems.}
\terminalcmd{sudo pacman -S gparted}
\block{Partition manager. Alternatively there is also \code{partitionmanager} (KDE)}

\subsection{Firewall}

Install \code{firewalld} and start the \code{firewalld.service}

If there is a systemd timeout \href{https://bugzilla.redhat.com/show_bug.cgi?id=1294415#c10}{issue} on restart, do:

\terminalcmd{sudo nano /etc/firewalld/firewalld.conf}
\begin{blocksection}
	Find the \code{CleanupOnExit} option in the file and set it to `\code{no}':\\
	\begin{codeblock}
		CleanupOnExit=no
	\end{codeblock}
\end{blocksection}

\subsection{Tools/Utils}

p7zip
unrar
Ark (GUI front end for compression)

\subsection{Dolphin extras}

\terminalcmd{sudo pacman -S kde-thumbnailer-odf kde-thumbnailer-epub kde-thumbnailer-gimpsources}
\block{Adds thumbnail support to ODF, EPUB and gimp source files. Enable in Dolphin's settings.}

\subsection{Fonts}

adobe-source-code-pro
adobe-source-sans-pro
adobe-source-serif-pro
ttf-mac-fonts

\subsection{Console apps}

`cmus`
`wget`

\subsubsection{Development}

valgrind, cmake, clang, extra-cmake-modules, clang-tools-extra, 
sqlitebrowser

\subsection{Desktop apps}

yakuake, thunderbird, firefox, cairo-dock, cairo-dock-plugins, 
deja-dup, truecrypt, veracrypt, 
redshift, plasma5-applets-redshift-control, 
discord

\clearpage
\section{Troubleshooting}

\subsection{Kernel errors}
\subsubsection{USB suspend error}

If you get the following sort of error after suspend/sleep:
\vspace*{0.6em}
\begin{codeblock}
	usb 2-4: usb\_reset\_and\_verify\_device Failed to disable LTM\linebreak	
	usb usb2-port4: cannot disable (err=-32)
\end{codeblock}

The Kernel \href{https://bugzilla.kernel.org/show_bug.cgi?id=117811}{Bug report \#117811}'s thread suggest disabling USB auto-suspend.

\begin{itemize}[noitemsep,topsep=0pt,leftmargin=*]
	\item If you are using \href{https://wiki.archlinux.org/index.php/TLP}{TLP}:\\
	\code{nano /etc/default/tlp} and set \code{USB\_AUTOSUSPEND=0}
	\item If you are \textbf{not} using \href{https://wiki.archlinux.org/index.php/TLP}{TLP}:\\
	\code{sudo echo on | tee /sys/bus/usb/devices/*/power/control}
\end{itemize}

\subsubsection{fail to get arp ip table err:-23}

//TODO `brcmfmac: brcmf\_inetaddr\_changed: fail to get arp ip table err:-23` after wifi connect

\subsubsection{kgd2kfd\_probe failure}

//TODO `kfd kfd: kgd2kfd\_probe failed` at start up

\subsection{Boot partition recognition is slow}

This is usually caused by booting a OSX drive externally or creating one with the recovery process (\fbox{Cmd}+\fbox{R}). 

\textbf{Beware:} Before going forward, make sure you have an \underline{bootable Arch installation USB drive} ready as you will also need to go reinstall GRUB as described in the next section.

To remedy this problem you will need to \href{https://support.apple.com/en-us/HT204063}{reset the NVRAM} by pressing \fbox{Alt}+\fbox{Command}+\fbox{P}+\fbox{R} together and turn on the machine keeping them pressed for about 20 seconds (or until the second boot-up chime).

\subsection{Boot partition not detected after PRAM reset}

Boot from the Arch setup USB (see section `Making a bootable USB').

Mount all the partitions again (section `7.3.3 Mount all the partition') without making the directories:

\terminalcmd{mount /dev/mapper/vg0-root /mnt}
\terminalcmd{mount /dev/mapper/vg0-home /mnt/home}
\terminalcmd{mount /dev/sda2 /mnt/boot}
\terminalcmd{mount /dev/sda1 /mnt/boot/efi}
\terminalcmd{swapon /dev/mapper/vg0-swap}

Get into the \code{arch-chroot} environment and reinstall GRUB:

\terminalcmd{arch-chroot /mnt /bin/bash}
\terminalcmd{run mkinitcpio -p linux}
\terminalcmd{run grub-install}
\terminalcmd{grub-mkconfig -o /boot/grub/grub.cfg}

Exit chroot environment and shutdown:

\terminalcmd{exit}
\terminalcmd{shutdown now}

Remove the USB stick and boot up.

\subsection{Brightness control still doesn't work}

\textbf{\textcolor{textgrey}{[Unconfirmed]}}

If you've patched the \code{apple-gmux} kernel module to no avail then installing the \href{https://github.com/0xbb/apple\_set_os.efi}{\code{apple\_set\_os.efi}} program in the GRUB bootloader chain might help jolt the system into recognising the brightness controls.

\clearpage
\section{References}

\begin{itemize}
	\item \href{https://wiki.archlinux.org/index.php/MacBookPro11,x#Using_the_MacBook.27s_native_EFI_bootloader_.28recommended.29}{Arch Linux's MacBookPro 11.x Wiki page}
	\item \href{https://gist.github.com/mattiaslundberg/8620837}{Mattias Lindberg's Encrypted volume Arch install guide}
	\item \href{https://www.howtoforge.com/tutorial/how-to-install-arch-linux-with-full-disk-encryption/}{HowToForge | How to install Arch Linux with Full Disk Encryption}
	\item \href{https://help.ubuntu.com/community/AppleKeyboard#Change_Function_Key_behavior}{Ubuntu | AppleKeyboard}
\end{itemize}

\end{document}
