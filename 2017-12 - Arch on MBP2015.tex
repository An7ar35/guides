% !TeX program = LuaLaTeX
\documentclass[12pt, a4paper]{article}
\usepackage[utf8]{inputenc} 
\usepackage[left=1cm,right=1cm,top=1.5cm,bottom=1.5cm,a4paper]{geometry}
\usepackage{fontspec} %Fonts
	\setmainfont{ClearSans-Regular.ttf}[
		Path		   = fonts/clear-sans/,
		BoldFont       = ClearSans-Bold.ttf,
		ItalicFont     = ClearSans-Italic.ttf,
		BoldItalicFont = ClearSans-BoldItalic.ttf
	]
	\setmonofont{LiberationMono-Regular.ttf}[
		Path 		   = fonts/liberation-mono/,
		BoldFont	   = LiberationMono-Bold.ttf,
		ItalicFont	   = LiberationMono-Italic.ttf,
		BoldItalicFont = LiberationMono-BoldItalic.ttf
	]
	\newfontfamily\openiconic{open-iconic}[Path = fonts/]
	\newfontfamily\devicon{devicon}[Path = fonts/]
	\newcommand{\icon}[2]{\raisebox{-0.175\height}{\textcolor{#1}{\openiconic{#2}}}}
\usepackage{ragged2e} %text alignment
\usepackage[table,xcdraw,usenames,dvipsnames]{xcolor}
	\definecolor{code-background}{HTML}{f3f3f3}
	\definecolor{dark}{HTML}{31363b}
	\definecolor{linegrey}{HTML}{a8abaf}
	\definecolor{textgrey}{HTML}{777777}
\usepackage{colortbl}
\usepackage{float}
\usepackage{graphicx}
\usepackage{tabularx}
\usepackage{verbatimbox}
\usepackage{enumitem}
	\newlist{itemize-steps}{itemize}{1}
	\setlist[itemize-steps,1]{label=,leftmargin=0cm}
\usepackage{url}
\usepackage{hyperref} % Hyperlink references
	\hypersetup{colorlinks=false,pdfborder=0 0 0}
\usepackage{mdframed} % minipage box
	\newmdenv{allborders}
	\newmdenv[topline=false,leftline=false,rightline=false,linecolor=linegrey]{bottomborder}
	\mdfdefinestyle{commentbox}{
		topline=false,
		rightline=false,
		bottomline=false,
		linewidth=1mm,
		linecolor=linegrey,
		splittopskip=0,
		splitbottomskip=0,
		frametitleaboveskip=0,
		frametitlebelowskip=0,
		skipabove=0,
		skipbelow=0,
		leftmargin=+0.1cm,
		rightmargin=0,
		innertopmargin=2mm,
		innerbottommargin=2mm,
		roundcorner=0mm,
		backgroundcolor=white
	}
	\newmdenv[topline=false,leftline=false,rightline=false,linecolor=dark,backgroundcolor=dark,fontcolor=white]{headerborder}
\usepackage{listings} %for code
\lstset{ %
	backgroundcolor=\color{code-background},
	basicstyle=\scriptsize,
	breakatwhitespace=false,         % sets if automatic breaks should only happen at whitespace
	breaklines=true,                 % sets automatic line breaking
	commentstyle=\color{grey},   	   % comment style
	frame=single,                    % adds a frame around the code
	keepspaces=true,                 % keeps spaces in text, useful for keeping indentation of code (possibly needs columns=flexible)
	keywordstyle=\color{Violet},       % keyword style
	%numbers=left,                    % where to put the line-numbers; possible values are (none, left, right)
	%numbersep=5pt,                   % how far the line-numbers are from the code
	%numberstyle=\tiny\color{gray}, % the style that is used for the line-numbers
	%rulecolor=\color{black},         % if not set, the frame-color may be changed on line-breaks within not-black text (e.g. comments (green here))
	%stepnumber=1,                    % the step between two line-numbers. If it's 1, each line will be numbered
	tabsize=4,
	title=\lstname                   % show the filename of files included with \lstinputlisting; also try caption instead of title
}
\newcommand{\code}[1]{ %Code text formatting
	\begin{small}
		\colorbox{code-background}{\texttt{#1}}
	\end{small}
}
\newcommand{\terminalcmd}[1]{ %Terminal line formatting
	\textcolor{orange}{\$} \code{#1}
}
\newcommand{\block}[1]{
	\begin{mdframed}[style=commentbox]
		\color{dark}{#1}
	\end{mdframed}
}
\color{dark}
\setlength{\parindent}{0em}
\setlength{\parskip}{0.5em}
\renewcommand{\baselinestretch}{1.2}

\begin{document}
\centering
\LARGE
Arch Linux installation guide for MacBook Pro Retina\\
 - mid-2015 model - 
\normalsize\justify
\tableofcontents
\clearpage

\section{Forewords}

\noindent After lots of reading, searching, experimenting, furious late-night shell command typing and 
do-overs here are the results of replacing OSX with Arch Linux (KDE) on a mid-2015 MacBook Pro. 

\noindent This guide is a record of my experience and results. I'm leaving this out there for anyone that
might find this useful.

\noindent I would also advise backing up your drive using a complete bit-to-bit cloning process (unlike me...oops) so that you retain a copy of everything including the recovery partition on the Mac. 
It's better to be safe...

\section{Machine Specs}

\begin{tabular}{c|l}
	\hline
	Display   & 15.4" LED-backlit Retina display with IPS technology; 2880-by-1800 native resolution at 220 pixels per inch with support for millions of colours \\\hline
	Processor & 2.5GHz quad-core Intel Core i7 processor (Turbo Boost up to 3.7GHz) with 6MB shared L3 cache \\\hline
	RAM       & 16GB of 1600MHz DDR3L memory \\\hline
	GPU       & Intel Iris Pro Graphics, <br>AMD Radeon R9 M370X with 2GB of GDDR5 memory \\\hline
	Storage   & 512GB PCIe-based flash \\\hline
	Webcam    & 720p FaceTime HD camera \\\hline
	Network   & 802.11ac Wi‑Fi wireless networking; IEEE 802.11a/b/g/n compatible, <br>Bluetooth 4.0 wireless technology \\\hline
\end{tabular}

\section{Results}

//TODO

//Heating issues with mbp, fan control and turning off boost on the processor to keep things
reasonable temperature-wise.

\section{Pre-Installation}

\subsection{Preparations}

\begin{itemize-steps}
	\item \textbf{Backup drive} (complete drive clone preferable).
	\item Make a copy of the colour profile file(s) on the mac located in \code{/Library/ColorSync/Profiles/Displays/*} to a USB stick. It will be useful later on Linux.

\end{itemize-steps}

\subsection{Making a bootable USB}

\subsubsection{From Linux}

\begin{itemize-steps}
	\item \terminalcmd{dd if=archlinux.iso of=/dev/sdX bs=16M \&\& sync} \\
	      \block{Where \code{X} is your target USB drive letter (use \code{lsblk} for an overview of all connected drives to find out)}
\end{itemize-steps}

\subsubsection{From a Mac}

\begin{itemize-steps}
	\item \terminalcmd{diskutil unmountDisk diskX} \\
		  \block{Where \code{X} is your target USB drive number (use \code{diskutil list} for an overview of all connected drives to find out).}
	\item \terminalcmd{sudo dd if=/Users/\$USERNAME/Downloads/archlinux.iso of=/dev/diskX bs=16M}\\
		  \block{Replace \code{\$USERNAME} with your username on the mac and replace \code{X} with the USB drive's number.}
\end{itemize-steps}


\section{Installation}

\subsection{Booting from the USB stick}

Simply plug in the USB in the MBP and press the \fbox{Alt} (a.k.a. \fbox{options}) key during start up
to reach the boot menu.

\subsection{Preliminary setup}

\begin{itemize-steps}
	\item \terminalcmd{loadkeys uk}\\
		  \block{Load your keyboard layout. Replace `uk` with whichever you have on your machine.}
	\item \terminalcmd{wifi-menu}\\
		  \block{Connect up to the wifi network. On the MPB Pro 2015 the wifi gets detected and works out of the box at this stage.}
\end{itemize-steps}


\clearpage
\section{References}

\begin{itemize}
	\item \href{https://wiki.archlinux.org/index.php/MacBookPro11,x#Using_the_MacBook.27s_native_EFI_bootloader_.28recommended.29}{Arch Linux's MacBookPro 11.x Wiki page}
	\item \href{https://gist.github.com/mattiaslundberg/8620837}{Mattias Lindberg's Encrypted volume Arch install guide}
	\item \href{https://www.howtoforge.com/tutorial/how-to-install-arch-linux-with-full-disk-encryption/}{HowToForge | How to install Arch Linux with Full Disk Encryption}
\end{itemize}

\end{document}